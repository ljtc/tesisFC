\documentclass{tesisFC}

%UN ESPACIO EN EL PREÁMBULO PARA PAQUETES
%_______________________________________________________________________________
%Ya se cargaron los paquetes mathtools (incluye amsmath), amsthm, siunitx, multicol, graphicx y xstring. Se recomienda fuertemente no volver a cargarlos.

%Estos son recomendados
\usepackage[backend=biber,defernumbers,style=alphabetic,giveninits=true]{biblatex}
\usepackage[style=mexican]{csquotes}

%este debe ser el último en casi todos los casos
\usepackage[colorlinks]{hyperref}

%_______________________________________________________________________________


%BIBLIOGRAFÍA CON BIBLATEX
%_______________________________________________________________________________
\DeclareBibliographyCategory{cited}
\AtEveryCitekey{\addtocategory{cited}{\thefield{entrykey}}}
\addbibresource{refs.bib}%cambiar por el nombre correcto
\nocite{*}
%_______________________________________________________________________________


%INFORMACIÓN PARA LA PORTADA
%_______________________________________________________________________________
\title{Título del trabajo}
\author{Nombre del alumno}
\grado{Grado a obtener}
\date{2021}
\tipo{tesis}
\tutorW{tutora}
\tutor{Grado y nombre}
%_______________________________________________________________________________


%AQUÍ PUEDEN IR LAS DEFINICIONES DE COMANDOS Y FUNCIONES
%_______________________________________________________________________________
%...
%_______________________________________________________________________________


\begin{document}
\frontmatter
\portadatesis
\thispagestyle{empty}
\begin{center}
  {\LARGE Hoja de datos del jurado}
\end{center}
\begin{multicols*}{2}
  \begin{enumerate}
    \item Datos del alumno\\
          ApellidoP\\
          ApellidoM\\
          Nombres\\
          teléfono\\
          Universidad Nacional\\
          Autónoma de México\\
          Facultad de Ciencias\\
          Carrera\\
          Cuenta
    \item Datos del tutor\\
          Grado\\
          Nombres\\
          ApellidoP\\
          ApellidoM
    \item Datos del sinodal 1\\
          Grado\\
          Nombres\\
          ApellidoP\\
          ApellidoM
    \item Datos del sinodal 2\\
          Grado\\
          Nombres\\
          ApellidoP\\
          ApellidoM
    \item Datos del sinodal 3\\
          Grado\\
          Nombres\\
          ApellidoP\\
          ApellidoM
    \item Datos del sinodal 4\\
          Grado\\
          Nombres\\
          ApellidoP\\
          ApellidoM
    \item Datos del trabajo escrito\\
          Título\\
          Subtítulo\\
          número de páginas\\
          año
  \end{enumerate}
\end{multicols*}
\cleardoublepage

\setcounter{page}{1}
\tableofcontents*

\mainmatter
%poner aquí el contenido principal de la tesis

\appendix
%aquí van lo apéndices

\backmatter
\printbibliography[title={Referencias},category=cited]
\printbibliography[title={Bibliografía},notcategory=cited]
\end{document}