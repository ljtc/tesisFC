%% !TEX root = guia.tex
%
%
%% arara: lualatex: { draft: yes } unless exists(toFile('ejemplo.aux'))
%% arara: biber if changed (toFile('refs.bib')) || found('log', 'Citation')
%% arara: --> || found ('log', 'Please \\(re\\)run Biber')
%% arara: makeglossaries if missing('gls')
%% arara: --> || changed('glo') || changed(toFile('gloss.tex'))
%% arara: lualatex: { synctex: yes } until !found('log', '\\(?(R|r)e\\)?run (to get|LaTeX)')

\documentclass{tesisFC}

%PAQUETES para los ejemplos
\usepackage{lipsum}%texto de demostración
\usepackage{xcolor}
\usepackage{tikz}
\usepackage[edges]{forest}
\usepackage[backend=biber,defernumbers,style=alphabetic,giveninits=true]{biblatex}%debe estar después del paquete de idiomas
\usepackage[style=mexican]{csquotes}%recomendado para usar junto con biblatex

%este sí es recomendado, pero debe ser el último en casi todos los casos
\usepackage[colorlinks]{hyperref}
%para quitar el color de los links sustituir colorlinks por hidelinks

%BIBLIOGRAFÍA
\DeclareBibliographyCategory{cited}
\AtEveryCitekey{\addtocategory{cited}{\thefield{entrykey}}}
\addbibresource{refs.bib}
\nocite{*}

%GLOSARIOS
\let\printindex\relax
\usepackage[symbols,index,abbreviations,automake]{glossaries-extra}
\newglossary*{nom}{Lista de Nombres}
\makeglossaries
\loadglsentries{gloss}

%INFORMACIÓN PARA LA PORTADA
\title{Título del trabajo}
\author{Nombre del alumno}
\grado{Grado a obtener}
\date{2021}
\tipo{tesis}
\tutorW{tutora}
\tutor{Grado y nombre}



%%%%%%%%%%%%%%%%%%%%%%%%%%%%%%%%%%%%%%%%%%%
%A partir de aquí se usa para los ejemplos%
%%%%%%%%%%%%%%%%%%%%%%%%%%%%%%%%%%%%%%%%%%%

%FLOTANTES
%subfloats en ambiente figura
\newsubfloat{figure}
%un nuevo flotante para diagramas con subflotantes
\newcommand{\diagramname}{Diagrama}
\newcommand{\listdiagramname}{Índice de Diagramas}
\newlistof{listofdiagrams}{dgm}{\listdiagramname}
\newfloat{diagram}{dgm}{\diagramname}
\newlistentry{diagram}{dgm}{0}
\newsubfloat{diagram}


%NUEVOS OPERADORES Y DELIMITADORES
\DeclareMathOperator{\idem}{Idem}
\DeclarePairedDelimiter{\abs}{\lvert}{\rvert}
\providecommand\given{}%por si no está definido
\newcommand\SetSymbol[1][]{%
  \nonscript\:#1\vert%
  \allowbreak%
  \nonscript\:
  \mathopen{}}
\DeclarePairedDelimiterX\Set[1]\{\}{%
\renewcommand\given{\SetSymbol[\delimsize]}#1}


%LIBRERÍAS Y CONFIGURACIÓN DE TIKZ
\usetikzlibrary{cd}
%lo ideal es que las flechas en los diagramas se vean igual
%a las del texto, tikz-cd puede hacer esto
\tikzcdset{arrow style=math font}
%lo malo es que en muchas fuentes hace que se "rompan" las flechas
%para que no se vean "rotas" hay que ajustar el ancho y línea de base
\tikzcdset{every arrow/.append style={line width=0.068em}}%ancho
\pgfmathdeclarefunction*{axis_height}{0}{%
    \begingroup%      
      \pgfmathreturn0.259em % línea de base
    \endgroup}% 
\usepackage{tkz-euclide}

%UN EJEMPLO DE TEOREMA CON FONDO DE COLOR
\definecolor{morado}{RGB}{101, 43, 145}
\newkeytheorem{teo}[
  name=Teorema,
  tcolorbox-no-titlebar={
    colback=morado!10, colframe=morado!10,
    left=5pt, right=5pt, top=2pt, bottom=2pt
  }
]


\ifluatex%
%Cargar fuentes adicionales con LuaLaTeX
\setmathfont{GFS Neohellenic Math}[version=sansmath]
\setsansfont{GFS Neohellenic}
\setmathfont{XITS Math}[range=scr]
\setmathfont{Erewhon Math}[range=\mscrS]

%Ambientes nuevos
\newcounter{sans}% contador para el ambiente
\NewDocumentEnvironment { sans }% nombre del ambiente
{ O{Este texto está en sans} }% título que imprime con un valor por defecto
{% código al inicio del ambiente
  \par\vspace{\baselineskip}\sffamily\mathversion{sansmath}\noindent
  \refstepcounter{sans}%
  \textbf{#1\space\thesans\space}\ignorespaces%
}
{%código al final
  \par\vspace{\baselineskip}%
  \noindent\ignorespacesafterend%
}
\fi
%este es el final del código para los ejemplos
%%%%%%%%%%%%%%%%%%%%%%%%%%%%%%%%%%%%%%%%%%%


\begin{document}

\frontmatter% parte frontal del texto, numeración en romanos
\portadatesis%
\thispagestyle{empty}

\begin{center}
  {\LARGE Hoja de datos del jurado}
\end{center}
\begin{multicols*}{2}
  \begin{enumerate}
    \item Datos del alumno\\
          ApellidoP\\
          ApellidoM\\
          Nombres\\
          teléfono\\
          Universidad Nacional\\
          Autónoma de México\\
          Facultad de Ciencias\\
          Carrera\\
          Cuenta
    \item Datos del tutor\\
          Grado\\
          Nombres\\
          ApellidoP\\
          ApellidoM
    \item Datos del sinodal 1\\
          Grado\\
          Nombres\\
          ApellidoP\\
          ApellidoM
    \item Datos del sinodal 2\\
          Grado\\
          Nombres\\
          ApellidoP\\
          ApellidoM
    \item Datos del sinodal 3\\
          Grado\\
          Nombres\\
          ApellidoP\\
          ApellidoM
    \item Datos del sinodal 4\\
          Grado\\
          Nombres\\
          ApellidoP\\
          ApellidoM
    \item Datos del trabajo escrito\\
          Título\\
          Subtítulo\\
          número de páginas\\
          año
  \end{enumerate}
\end{multicols*}
\cleardoublepage



%Tablas de contenidos
\setcounter{page}{1}
\tableofcontents*
%\clearpage% \cleardoublepage para que empiece en página impar
%\listoffigures%
%\clearpage% \cleardoublepage para que empiece en página impar
%\listofdiagrams%


\mainmatter% parte principal del texto, numeración en arábigos
\include{minimo}
% !TEX root = ejemplo.tex

\chapter{Un poco de memoir}


\section{Lo básico}
La clase \texttt{tesisFC} tiene como base a \textit{memoir}. La motivación es poder configurar de manera fácil los aspectos del documento sin tener que usar paquetería adicional. Así, cuando el usuario cargue un paquete es poco probable que cause alguna incompatibilidad.

\textit{Memoir} es una clase configurable, es decir, no hace falta cargar paquetes adicionales para hacer un diseño completo de cómo se verá nuestro
documento. Además puede servir tanto para \textit{book} como para \textit{article} la opción por defecto es una salida estilo \textit{book}, para cambiar al diseño estilo \textit{article} basta poner la opción \texttt{article} como opción a la clase. A diferencia de las clases
estándar donde hacer un cambio de \textit{book} a \textit{article} seguramente causará problemas (por ejemplo \verb|\chapter{...}| no está
definido en \textit{article}), en \textit{memoir} no existe ese problema. Otra ventaja inmediata es que el ambiente \texttt{abstract} estará definido para \textit{book}.

La tabla de contenidos esta formada por el comando
\verb|\tableofcontents*|. Esta versión con estrella es única de \textit{memoir} y su utilidad es que no aparezca una entrada para la tabla de contenidos en la tabla de contenidos.

Otra ventaja de \textit{memoir} es que no hace falta redefinir
\verb|\cleardoublepage| para que las páginas pares \enquote{vacías} antes de un
nuevo capítulo estén en blanco. La clase ya lo hace por defecto. Además, la
clase tiene las opciones \texttt{openright} para que los capítulos empiecen
en las paginas impares, \texttt{openleft} para que empiecen en las pares y
\texttt{openany} para que empiecen en la siguiente página sin importar su
paridad.

\textit{Memoir} tiene predefinidos muchos estilos de capítulo, en este documento se está usando \texttt{madsen}. Además tiene
definidos muchos más estilos de salida por defecto, estos pueden verse en
\url{http://www.ebookation.com/wp-content/uploads/2010/03/memoirchapstyles.pdf}

Modificamos el tamaño y la posición de la caja de texto para obtener una linea de texto suficientemente grande como evitar tantos cortes de palabra, pero no tan grande como para que sea difícil pasar de una linea a otra.

Si notan como está formado un libro, el bloque de texto no está centrado
en la página. Comúnmente el margen de la espina (donde se juntan las
páginas de un libro) es la mitad del margen de la orilla (el opuesto a
la espina). Lo mismo sucede con los margenes superior e inferior, donde el
superior es más chico que el inferior. Tomamos en cuenta estos detalles en
la formación de este documento.

También cambiamos cabeceras y pies, esta vez fue mínimo pero aún así es
suficiente para ver cómo cambiar cabeceras y pies (ver archivo \texttt{.cls}).

Mencionamos de nuevo que \textit{memoir} es una clase configurable y tiene de
manera nativa capacidades mayores o iguales a
las de los siguientes paquetes:
\begin{center}
  \autocols{c}{6}{l}{abstract, appendix, booktabs, ccaption, chngcntr, chngpage, enumerate, epigraph, framed, ifmtarg, index, makeidx, moreverb, needspace, newfile, nextpage, parskip, patchcmd, setspace, shortvrb, showidx, titleref, titling, tocbibind, tocloft, verbatim, verse}%
\end{center}%
También tiene las capacidades, aunque con comandos diferentes, de los siguientes paquetes
\begin{center}
  crop, fancyhdr, geometry, sidecap, subfigure, titlesec.
\end{center}
Por último, la clase carga los siguientes paquetes
\begin{center}
  array, dcolumn, delarray, etex, iftex, tabularx, textcase (con la opción \texttt{overload})
\end{center}
Así que no hace falta cargar ninguno de estos, de esta manera es más fácil
tener compatibilidad de paquetes.

\section{Un poco más}
Primero veamos cómo hacer subfiguras con memoir, es decir, sin usar el paquete
\texttt{subfigure} o \texttt{subcaption}. Primero necesitamos escribir
\begin{flushleft}
  \verb|\newsubfloat{figure}|
\end{flushleft}
en el preámbulo. Este comando activará todo lo necesario para hacer
subfiguras, por ejemplo en la figura~\ref{fig:vect} hay
una forma de hacerlas usando el comando \verb|\subcaption{...}| para poner
los subtítulos correspondientes. También es posible hacer como en el
ejemplo de la figura~\ref{fig:muchas}.
\begin{figure}
  \centering
  \subbottom[Primera]{%
    \includegraphics[width=0.3\linewidth]{example-image-a}}
  \subbottom[Segunda]{%
    \includegraphics[width=0.3\linewidth]{example-image-b}}
  \subbottom[Tercera]{%
    \includegraphics[width=0.3\linewidth]{example-image-c}}
  \caption{Muchas figuras}%
  \label{fig:muchas}
\end{figure}
Nota que no hay archivos aparte para las figuras que usamos. Estas ya están
instaladas con nuestra distribución y podemos usarlas como lo hicimos.
Además como su nombre lo indica, un \textit{float} está flotando en la
página y \LaTeX{} decidirá cual es el mejor lugar para ponerlo. Otro ejemplo
de flotante es una tabla. Las opciones para la ubicación de flotantes son
las siguientes.
\begin{description}
  \item [h] trata de poner el flotante donde fue creado.
  \item [t] lo pone en la parte superior de la página.
  \item [b] lo pone en la parte inferior de la página.
  \item [p] lo pone en una página especial para flotantes.
  \item [h!] se esfuerza más en poner el flotante donde fue creado.
\end{description}
La sintaxis para especificar la posición es \verb|\begin{figure}[...]|.

También es posible crear nuevos flotantes, nosotros hemos hecho un flotante
para diagramas en el preámbulo. Así podemos poner diagramas en una versión
similar a la de figura.
\begin{diagram}
\centering
\begin{tikzpicture}[commutative diagrams/every diagram]
  \node (P0) at (90:2.3cm) {\(X\otimes (Y\otimes (Z\otimes T))\)};
  \node (P1) at (90+72:2cm) {\(X\otimes ((Y\otimes Z)\otimes T))\)} ;
  \node (P2) at (90+2*72:2cm) {\makebox[5ex][r]{\((X\otimes (Y\otimes Z))\otimes T\)}};
  \node (P3) at (90+3*72:2cm) {\makebox[5ex][l]{\(((X\otimes Y)\otimes Z)\otimes T\)}};
  \node (P4) at (90+4*72:2cm) {\((X\otimes Y)\otimes (Z\otimes T)\)};
  \path[commutative diagrams/.cd, every arrow, every label]
    (P0) edge node[swap] {\(1\otimes\alpha\)} (P1)
    (P1) edge node[swap] {\(\alpha\)} (P2)
    (P2) edge node[swap] {\(\alpha\otimes 1\)} (P3)
    (P4) edge node {\(\alpha\)} (P3)
    (P0) edge node {\(\alpha\)} (P4);
\end{tikzpicture}
\caption{¡Pentagonator!}
\end{diagram}
También es posible aprovechar el espacio del margen como apoyo para las figuras. Esto es recomendable sólo en el caso cuando haya suficiente espacio en el margen. Por ejemplo, en la versión para imprimir no hay espacio en el margen, así que, con la opción \texttt{imp} se redefinió el ambiente \texttt{marginfigure} para que se convierta en un \texttt{figure}.
\begin{marginfigure}
\centering
  \begin{tikzpicture}
    \draw (0,0) circle (5mm);
  \end{tikzpicture}
  \caption{Un circulo en el margen}
\end{marginfigure}

En la versión para imprimir la tesis como un libro el margen se
reduce para permitir una caja de texto más grande, así las cosas que estén
en el margen podrían verse muy \enquote{apretadas}. Por esta razón se quitaron automáticamente todas las figuras del margen.


Por último en el código de este documento se puede ver que todas las
figuras tienen inmediatamente después del inicio del ambiente el comando
\verb|\centering|, debe ser este y no usar el ambiente \texttt{center}
para evitar espacios verticales no deseados. Además, como esto se hizo en
cada figura pudo haberse configurado en el preámbulo con el siguiente
comando
\begin{flushleft}
  \verb|\setfloatadjustment{figure}{\centering}|
\end{flushleft}
que también se puede usar en tablas, figuras al margen y los flotantes que
se hayan creado.

\include{fuentes}
% !TEX root = guia.tex

\chapter{Paquetes en la clase}

\section{amsmath y mathtools}
Aunque en el título de la sección se menciona \texttt{amsmath} en la clase
sólo se carga el paquete \texttt{mathtools}. Esto se debe a que
\texttt{mathtools} carga al paquete \texttt{amsmath}, así lo podemos pensar
como una extensión. De manera más precisa \texttt{mathtools} es resultado de
corregir algunos bugs de \texttt{amsmath} y añadir algunas otras funciones.

Un ejemplo específico de estos paquetes es la creación de operadores. En el
modo matemático de \LaTeX{} una letra representa ua variable y dos variables
juntas representa la multiplicación de estas. De esta manera \(xy\,a\)
representa la multiplicación de las variables \(x\), \(y\) y \(a\), mientra
que \(\operatorname{xy}a\) representa al operador \(\operatorname{xy}\)
aplicado a la variable \(a\). En otras palabras, un cambio en el alfabeto
debería implicar un cambio en el significado. Por esta razón deberíamos
escribir operadores de una forma especial y no sólo escribiendo su nombre en
modo matemático.

Por ejemplo, para escribir el supremo del conjunto \(A\) se debe escribir
\verb|\sup A| y su resultado es \(\sup A\), de donde es claro que los
operadores están en letras \textit{upright} (redondas en español). Ya están
definidos los operadores más comunes, pero si se quisiera definir uno nuevo
se escribe en el preámbulo \verb|\DeclareMathOperator{\idem}{Idem}| para
definir, por ejemplo, los idempotentes de un anillo, así \verb|\idem(R)|
genera \(\idem(R)\).

Además \texttt{amsmath} define ambientes matemáticos como
\texttt{align}, \texttt{gather}, \texttt{cases}, \texttt{equation},
\texttt{array}, las variantes de \texttt{matrix}, etc.\ y \texttt{mathtools}
corrige algunos errores en ellos, por ejemplo en \texttt{gather} o añade
algunas opciones, por ejemplo en los ambientes de matriz se puede
especificar la alineación de las columnas de la misma manera que en
\texttt{array}.

Una función de \texttt{mathtools} que no tiene \texttt{amsmath} es la creación
de delimitadores que se pueden ajustar al tamaño del contenido, por ejemplo para
hacer un delimitador para el valor absoluto se escribe
\verb|\DeclarePairedDelimiter\abs{\lvert}{\rvert}| este tiene un argumento más
que el operador ya que hay que decir qué símbolo \enquote{abre} y que símbolo
\enquote{cierra}. No es necesario que concuerde un símbolo que abre con su
pareja de cierre, como los paréntesis. Por ejemplo se podría definir un
delimitador para intervalos abiertos por la izquierda y cerrados por la derecha.
La deferencia de estos delimitadores se puede apreciar con su versión con
estrella y con el código

\begin{tabular}{llll}
  \verb+|\sum_{i=1}^{n}a_1|+ & \(|\sum\limits_{i=1}^{n}a_1|\), &
  \verb|\lvert\sum_{i=1}^{n}a_1\rvert| &
  \(\lvert\sum\limits_{i=1}^{n}a_1\rvert \) \\[4ex]
  \verb|\abs{\sum_{i=1}^{n}a_1}| & \(\abs{\sum\limits_{i=1}^{n}a_1}\), &
  \verb|\abs*{\sum_{i=1}^{n}a_1}| & \(\abs*{\sum\limits_{i=1}^{n}a_1}\)
\end{tabular}

Más aún, no se debe usar las barras comunes para el valor absoluto como en
\verb+|x|+ ya que no tienen el contexto adecuado. Cuando se definen como
delimitadores, \texttt{mathtools} les agrega un \verb|\mathopen{}| y
\verb|\mathclose{}| como corresponde. La función de los dos comandos anteriores
se puede apreciaren el siguiente ejemplo

\begin{tabular}{llll}
  \verb+|x|+ & \(|x|\) & \verb¡|-x|¡ & \(|-x|\)\\
  \verb+\abs{x}+ & \(\abs{x}\) & \verb+\abs{-x}+ & \(\abs{-x}\),
\end{tabular}

\noindent donde claramente el espaciado en \verb¡|-x|¡ es pésimo. El mal
espaciado se debe a que \LaTeX{} reconoce a \verb|-| como un operador binario y
da un poco de espacio entre los símbolos a su derecha e izquierda. En cambio,
cuando se le ha agregado \verb|\mathopen{}| al delimitador, como sucede en
nuestra definición de \verb|\abs|, \LaTeX{} ahora toma a \verb|-| como operador
unario y no agrega espacio a la izquierda.

\verb|\DeclarePairedDelimiter| tiene una versión extendida para que pueda
tomar argumentos, como \verb|\newcommand{}[]{}|. Esta versión extendida se
escribe \verb|\DeclarePairedDelimiterX| y con ella podemos hacer, por
ejemplo, conjuntos  con delimitadores y la línea de \enquote{tal que} que
crezcan correctamente (ver su definición en \texttt{ejemplo.tex})
\[
  \Set*{ x \in X \given \frac{\sqrt{x}}{x^2+1} > 1 }
\]


\section{keytheorems}
El paquete \texttt{keytheorems} es una extensión del paquete \texttt{amsthm} que
permite definir teoremas, lemas, proposiciones, etc. de una manera más sencilla.
Además, también extiende y corrige errores del paquete \texttt{thmtools}. Se
eligió este paquete por ser sencillo y tener mayor soporte que dos paquetes
juntos.

En la clase se definieron los ambientes más usuales de \enquote{teoremas}. Se
definieron usando su mismo nombre en español, incluyendo acentos, para evitar
posibles conflictos con otros paquetes. Por ejemplo, podemos usar definiciones.

\begin{definición}[label=test]
  Esta es una definición de ejemplo.  
\end{definición}

Luego, se puede hacer referencia a ella. Por ejemplo, en la
definición~\ref{test} vimos un ejemplo. Luego tenemos algunos otros ejemplos:

\begin{lema}
  \lipsum[1][1]
\end{lema}

\begin{teorema}[store=teo]
  \lipsum[1][2]
\end{teorema}

\begin{corolario}
  \lipsum[1][3]
\end{corolario}

\begin{proposición}
  \lipsum[1][4]
\end{proposición}

\begin{observación}[store=obs]
  \IfRestatingT{Volvamos a notar que:} \lipsum[1][5]
\end{observación}

En el último de los ejemplos definimos un \enquote{teorema} nuevo para escribir
axiomas. Revisar el preámbulo de esta guía para ver cómo se definió.

\begin{ax}[note=Peano]
  \(0\in\mathbb{N}\).
\end{ax}

Luego, el paquete tiene facilidades para volver a enunciar teoremas.

\getkeytheorem{teo}

También tiene booleanos para escribir algo diferente en la versión que se volvió
a enunciar.

\getkeytheorem{obs}

Además, se puede hacer una lista de teoremas.

\listofkeytheorems

Sólo se ha mostrado lo más básico del paquete. Se recomienda ver la
documentación para ver todas las posibilidades que tiene.


\section{babel y polygossia}%
\label{sec:babel}
Para el soporte de idiomas elegimos \texttt{babel} ya que tiene mayor soporte que otros paquetes similares, como \texttt{polygossia}.

\texttt{babel} se carga con las opciones \texttt{spanish} y
\texttt{mexico}. La opción \texttt{mexico} hace una localización más similar
a la que se usa en México, como su nombre lo indica. Entre las cosas que
hace podemos mencionar que cambia el nombre \enquote{cuadro} por \enquote{tabla},
prioriza comillas y usa el punto decimal en lugar de la coma como puede
verse en la compilación con \(\pi=3.141592\ldots \) Además se
cargó la opción \texttt{es-noindentfirst} para evitar el sangrado en el
primer párrafo después de un título. Por
último se deshabilitaron los \texttt{shorthands} o taquigrafías para evitar
algunos conflictos con la sintaxis de otros paquetes, por ejemplo las
comillas en \texttt{xy-pic} o en \texttt{tikzcd}.

Al usar el idioma español \texttt{babel} se encargará de traducir todo, por
ejemplo la palabra \enquote{capítulo} o \enquote{figura} como se puede ver
en el documento. También traduce y acentúa los operadores, por ejemplo
\(\max A\) o \(\lim f\). Pero en algunos casos se decidió crear un comando
nuevo como en el caso del seno:
\[
  \sin(\alpha)\ne\sen(\alpha)
\]


Otra opción para el manejo de idiomas en Xe\LaTeX{} o Lua\LaTeX{} es
\texttt{polyglossia}. Este paquete se creó cuando \texttt{babel} había
dejado de tener mantenimiento y uno de sus objetivos es simplificar su
trabajo en estos motores. Este paquete también tiene una variante
\texttt{mexico} (no compatible con \texttt{biber}), además de que se puede
elegir un poco más el comportamiento de los operadores con las opciones:
\begin{description}
  \item[\texttt{accented}] para acentuar los operadores como \texttt{babel}.
  \item[\texttt{spaced}] para dar un espacio corto entre el operador y el objeto al que se le aplica, de nuevo, como \texttt{babel}.
  \item[\texttt{all}] para hacer las dos anteriores más la localización de \verb|\sin| \verb|\tan| \verb|\sinh| y \verb|\tanh|.
  \item[\texttt{none}] para no hacer ninguna localización.
\end{description}
Con Lua\LaTeX{} se podría cargar \texttt{polygossia} con la siguiente
configuración
\begin{flushleft}
  \verb|\usepackage{polyglossia}| \\
  \verb|\setdefaultlanguage[spanishoperators=all]{spanish}|
\end{flushleft}

De nuevo, para evitar el sangrado del primer párrafo después de un título
se debería añadir el comando
\begin{flushleft}
  \verb|\PolyglossiaSetup{spanish}{indentfirst=false}|
\end{flushleft}

Tanto \texttt{babel} como \texttt{polyglossia} son buenas opciones para el
manejo de idiomas y ambos tienen ventajas y desventajas en algunos aspectos
sobre el otro. Aún así elegimos usar \texttt{babel} por ser más conocido y tener un mejor soporte.


\section{microtype}
Este paquete habilita los aspectos micro-tipográficos del documento. Algunos
de ellos ya son hechos por \TeX{} como justificación y la separación
silábica (cortes de palabra). En general las opciones que se le pasaron a
\texttt{microtype} sirven para calcular la expansión de letras, palabras y
permitir que algunos caracteres cortos, como el guión de un corte de palabra o
el punto, salgan del margen. Algunas de estas mejoras se pueden hacer con
\texttt{fontspec}, por ejemplo poniendo la opción \texttt{LetterSpacing} al
cargar una fuente, pero hay que calcular las cosas manualmente. En cambio
\texttt{microtype} toma todas estas decisiones por nosotros y lo hace
de acuerdo al idioma que se la haya pasado a \texttt{babel} o
\texttt{polyglossia}. El resultado es un pdf que luce tipográficamente mejor.


\section{siunitx}
Este paquete tiene como objetivo implementar la escritura de cantidades
físicas de acuerdo con las reglas del sistema internacional de medidas.\footnote{\url{https://www.bipm.org/en/measurement-units/}} Aunque hay un desacuerdo con la forma de espaciar las cantidades y las unidades debido a la mala traducción del francés (idioma de la versión oficial del manual del sistema internacional de medidas) al inglés (idioma de donde se basaron para la creación del paquete). En el francés dice que deberían separase con un espacio y en el inglés dice que se separan con un \textit{thin space}. Así, el espaciado no es el correcto pero el error puede ser productivo ya que un menor espaciado crea una relación más estrecha entre cantidad y unidad. Algunos ejemplos de este paquete están en la siguiente tabla:
\begin{center}
  \begin{tabular}{lc}
    \toprule
    Comando & Resultado \\
    \midrule
    \verb|\ang{1;2;3}| & \ang{1;2;3}\\
    \verb|\unit{\gram\per\cubic\centi\metre}| & \unit{\gram\per\cubic\centi\metre}\\
    \verb|\unit{kg.m/s^2}| & \unit{kg.m/s^2}\\
    \verb|\qty{.23e7}{\candela}| & \qty{.23e7}{\candela} \\
    \verb|\qty[mode=text]{1.23}{J.mol^{-1}.K^{-1}}| & \qty[mode=text]{1.23}{J.mol^{-1}.K^{-1}} \\
    \verb|\qty[per-mode = symbol]{1.99}{\per\kilogram}| & \qty[per-mode = symbol]{1.99}{\per\kilogram} \\
    \verb|\qty[per-mode=fraction]{1,345}{\coulomb\per\mole}| & \qty[per-mode=fraction]{1,345}{\coulomb\per\mole}\\
    \bottomrule
  \end{tabular}
\end{center}


\section{csquotes}%
\label{se:csq}
Este paquete no fue cargado por defecto en la clase pero es recomendado
para la bibliografía, sobretodo si se usa \texttt{biblatex}, como en nuestro
ejemplo.

Este paquete provee algunas facilidades en citas y ajustar las
comillas al idioma que se quiera. En este documento se uso junto con la
opción \texttt{style=mexican} que carga el estilo de comillas del idioma
español con la variante para México. También define comandos para poner
comillas y hacer citas, por ejemplo \verb|\enquote{comillas}| resulta en
\enquote{comillas}. Este comando se puede anidar para poner las comillas
correctas dentro de otras comillas, por ejemplo
\begin{flushleft}
  \verb|\enquote{Lorem ipsum \enquote{dolor} sit amet}|
\end{flushleft}
resulta en \enquote{Lorem ipsum \enquote{dolor} sit amet}. Puede ser útil
escribir comillas mediante un comando, ya que muchas veces no se usan las
comillas correctas de \LaTeX, estas son \verb|``...''|. Además, da
facilidades para poner citas textuales con el comando
\verb|\textquote[⟨cita⟩][⟨puntuación⟩]{⟨texto⟩}|. También tiene un comando
para citar un bloque de texto
\verb|\blockquote[⟨cita⟩][⟨puntuación⟩]{⟨texto⟩}|. Todas estas funciones
tienen una versión para citar en un idioma extranjero, este idioma también
debe ser cargado en \texttt{babel} (o \texttt{polygossia}).


\section{biblatex}%
\label{sec:bib}
En principio no se cargo por defecto ningún paquete para crear la
bibliografía del documento, pero el que nos parece recomendable es
\texttt{biblatex}. Hay otros paquetes para formar la bibliografía de un
texto, por ejemplo \texttt{natbib} o \texttt{cite}.

Mientras que \texttt{natbib} y \texttt{cite} son muy parecidos, sólo cambian
algunas capacidades y \texttt{natbib} puede hacer técnicamente todo lo que
hace \texttt{cite} más algunas cuantas cosas; \texttt{biblatex} es muy
diferente a estos dos paquetes.

Como este es un texto \enquote{local} en el sentido que su código fuente no
debe cumplir los requerimientos de ningún \textit{journal} o editorial,
entonces es posible usar los paquetes que cada quien considere necesario. La
motivación principal para hacer este ejemplo con \texttt{biblatex} está
en la siguente liga \url{https://tex.stackexchange.com/a/25702/140456},
donde se explican las ventajas de biber sobre bibtex.

La construcción de la base de datos de referencias es básicamente la misma
para cualquier paquete de los anteriores. Hay herramientas gráficas útiles
para hacer esto como JabRef o Zotero.

El paquete \texttt{biblatex} puede manejar muchos tipos de entrada. En el
manual se describe el uso de los siguientes
\autocols{c}{5}{l}{%
  article, book, mvbook, inbook, bookinbook, suppbook, booklet, collection, mvcollection, incollection, suppcollection, dataset, manual, misc, online, patent, periodical, suppperiodical, proceedings, mvproceedings, inproceedings, reference, mvreference, inreference, report, set, software, thesis, unpublished, xdata, custom[a-f], conference, masterthesis, pdhthesis, techreport, www, artwork, audio, bibnote, commentary, image, jurisdiction, legislation, legal, letter, movie, music, performance, review, standard, video
  }
Son demasiados para dar una descripción breve, pero en el manual de
\texttt{biblatex} no sólo se describe para qué se usa cada una sino que
también algunos campos recomendados para algunas de ellas.

La lista de campos que se pueden usar en las entradas es demasiado larga para
escribirla aquí. De nuevo, en el manual se dan los campos disponibles junto
con una descripción breve.

También tiene una lista grade de formas de citación, las más comunes son
\verb|\cite{...}|, \verb|\parencite{...}|, \verb|\textcite{...}| y
\verb|\footcite{...}|. El estilo de las entradas bibliográficas y de la
citación se hacen mediante opciones del paquete. Como la lista de opciones
también es amplia haremos referencia a la siguiente liga
\url{http://tug.ctan.org/info/biblatex-cheatsheet/biblatex-cheatsheet.pdf}
para tener una guía rápida de \texttt{biblatex}.

En este ejemplo hicimos un archivo \texttt{refs.bib} como base de datos de la
bibliografía. Para \enquote{cargarlo} hay que usar el comando
\verb|\addbibresource{refs.bib}|. Este comando sólo acepta un archivo, pero
se pueden cargar más escribiendo todo el comando con cada archivo
\texttt{.bib} que se quiera usar. Para imprimir la bibliografía se pone el
comando \verb|\printbibliography| donde se quiera tener la bibliografía. Por
ejemplo se puede imprimir una bibliografía por capitulo, por tipo
(artículo, libro, etc.), por alguna palabra clave, etc.

En principio sólo imprime las entradas que fueron citadas en el texto, si se
quiere imprimir una entrada que no fue citada se usa el comando
\verb|\nocite{label1,label2,...}| o si se quiere imprimir todas las entradas
del archivo \texttt{.bib} se usa \verb|\nocite{*}|. 

Como en el instructivo de tesis digital de la facultad de ciencias se pide 
hacer dos \enquote{bibliografías}, entonces se ha incluido \verb|\nocite{*}| en 
el preámbulo.

El paquete \texttt{biblatex} tiene muchas capacidades para hacer e imprimir 
bibliografías. Por ejemplo podría imprimir sólo los libros en una bibliografía 
y los artículos en otra. También es posible hacer una bibliografía por capítulo 
y muchas otras cosas más.

La facultad requiere una \enquote{bibliografia} donde estén todas las 
referencias que sí fueron citadas en el texto, que vaya con el título 
\enquote{Referencias}, y otra con todas aquellas que no fueron citadas, bajo el 
nombre \enquote{Bibliografía}. Como este es el requerimiento de la facultad, 
sólo incluimos un ejemplo de cómo hacer eso y nada más.

Primero se hizo una categoría para \texttt{biblatex} y se incluyeron en ella todos las referencias citadas. Para que la bibliografía con las referencias no citadas no aparezca vacía se necesita usar \verb|\nocite{*}|. Así, en el preámbulo del documento aparecen los comandos
\begin{verbatim}
\DeclareBibliographyCategory{cited}
\AtEveryCitekey{\addtocategory{cited}{\thefield{entrykey}}}
\addbibresource{refs.bib}
\nocite{*}
\end{verbatim}
Luego, al final del documento se incluyen los comandos para imprimir las dos bibliografías. Estos son
\begin{verbatim}
\printbibliography[title={Referencias},category=cited]
\printbibliography[title={Bibliografía},notcategory=cited]
\end{verbatim}

Para ver cómo funciona todo esto en el pdf llamado \texttt{guia.pdf} citamos algunas referencias,~\cite{ACohesion,Homotopy,ToposT,CWM}.

Una nota final es que el proceso de compilación ahora debe ser el siguiente
\begin{flushleft}
\verb|lualatex MiDocumento.tex|\\
\verb|biber MiDocumento|\\
\verb|lualatex MiDocumento.tex|\\
\verb|lualatex MiDocumento.tex|
\end{flushleft}

\include{glosarios}
\include{comp}

%\appendix %aquí van los apéndices, pero en este ejemplo no hay

\backmatter% parte posterior del texto
%Impresión de glosarios
\printindex[title=Índice de términos]
\printsymbols[title=Lista de símbolos]
\printabbreviations[title=Acrónimos]
\printglossary[type=nom]
%
%%Impresión de bibliografía
\printbibliography[title={Referencias},category=cited]
\printbibliography[title={Bibliografía},notcategory=cited]

\end{document}
